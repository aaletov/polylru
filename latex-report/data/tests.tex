\begin{itemize}
	\item ЕСЛИ конструктор вызывается с отрицательным параметром size, ТО выбрасывается исключение std::length\_error
	\item ЕСЛИ конструктор вызывается с параметром size равным 0, ТО выбрасывается исключение std::length\_error
	\item ЕСЛИ конструктор вызывается с положительным параметром size, ТО создаётся новый объект
	\item ЕСЛИ ёмкость кэша равна 1 И элемент с ключом k1 не присутствует в кэше И метод contains вызывается с ключом k1, ТО метод возвращает false
	\item ЕСЛИ ёмкость кэша равна 1 И элемент с ключом k1 присутствует в кэше И метод contains вызывается с ключом k1 ТО метод возвращает TRUE
	\item ЕСЛИ метод get вызывается с неизвестным ключом, ТО поведение не определено
	\item ЕСЛИ ёмкость кэша равна 1 И элемент с ключом k1 и значением v1 присутствует в кэше И метод get вызывается с ключом k1, ТО метод возвращает v1
	\item ЕСЛИ ёмкость кэша равна 1 И кэш заполнен И метод set вызывается с новым ключом ТО из кэша вытесняется самый старый элемент
	\item ЕСЛИ ёмкость кэша равна 1 И в кэше присутствует элемент с ключом k1 и значением v1 И вызывается метод set с ключом k1 и значением v2, ТО v2 перезаписывает v1
	\item ЕСЛИ ёмкость кэша >1 И кэш не заполнен И вызывается метод set с новым ключом, ТО элемент сохраняется в кэше
	\item ЕСЛИ ёмкость кэша >1 И в кэше присутствует элемент с ключом k1 и значением v1 И вызывается метод set с ключом k1 и значением v2, ТО v2 перезаписывает v1
	\item ЕСЛИ ёмкость кэша >1 И кэш заполнен И элемент с ключом k1 не присутсвует в кэше И вызывается метод set с ключом k1 и значением v1, ТО новый элемент вымещает самый старый элемент
	\item ЕСЛИ объект был инициализирован с параметром capacity равном x И вызывается метод capacity, ТО метод возвращает x
	\item ЕСЛИ в кэш было добавлено n элементов с разными ключами И n <= capacity И вызывается метод size, ТО метод возвращает n
	\item ЕСЛИ в кэш было добавлено n элементов с разными ключами И n > capacity ИИ вызывается метод size, ТО метод возвращает значение, равное capacity
	\item ЕСЛИ x элементов находится в кэше с ёмкостью y И вызывается метод resize с параметром z > y, ТО все элементы копируются в кэш с новым размером И значение capacity меняется на z
	\item ЕСЛИ x элементов находится в кэше с ёмкостью y И вызывается метод resize с параметром z < y, ТО y - z старших элементов вымещаются И z младших элементов копируются в кэш с новым размером И значение capacity меняется на z
\end{itemize}

